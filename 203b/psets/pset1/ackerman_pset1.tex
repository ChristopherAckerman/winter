% Created 2021-01-20 Wed 11:59
% Intended LaTeX compiler: pdflatex
\documentclass[11pt]{article}
\usepackage[utf8]{inputenc}
\usepackage[T1]{fontenc}
\usepackage{graphicx}
\usepackage{grffile}
\usepackage{longtable}
\usepackage{wrapfig}
\usepackage{rotating}
\usepackage[normalem]{ulem}
\usepackage{amsmath}
\usepackage{textcomp}
\usepackage{amssymb}
\usepackage{capt-of}
\usepackage{hyperref}
\usepackage{amsthm}
\usepackage{url}
\usepackage[margin=1.25in]{geometry}
\usepackage{hyperref}
\usepackage[dvipsnames]{xcolor}
\usepackage{booktabs}
\usepackage{enumitem}
\usepackage{minted}
\newtheorem*{definition}{Definition}
\newtheorem*{example}{Example}
\newtheorem*{theorem}{Theorem}
\newtheorem*{corollary}{Corollary}
\newtheorem*{exercise}{Exercise}
\newtheorem*{problem}{Problem}
\newtheorem{question}{Question}
\newcommand{\gr}{\textcolor{ForestGreen}}
\newcommand{\rd}{\textcolor{red}}
\newcommand{\R}{\mathbb{R}}
\newcommand{\p}{\mathbb{P}}
\newcommand{\E}{\mathbb{E}}
\newcommand{\inv}{^{-1}}
\newcommand{\frall}{\ \forall}
\author{Chris Ackerman\thanks{I worked on this problem set with Luna Shen.}}
\date{\today}
\title{Econ203B HW1}
\hypersetup{
 pdfauthor={Chris Ackerman\thanks{I worked on this problem set with Luna Shen.}},
 pdftitle={Econ203B HW1},
 pdfkeywords={},
 pdfsubject={},
 pdfcreator={Emacs 28.0.50 (Org mode 9.3)}, 
 pdflang={English}}
\begin{document}

\maketitle
\tableofcontents

\newpage

\section{Question 1}
\label{sec:org859546f}
 \begin{align*}
\intertext{In our sample, we have the minimization problem}
\beta_0 &= \arg \min_{b \in \R^2} \E [(Y - (1, X)b)^2]\\
\implies \beta_0 &= \E[(1, X)'(1, X)]\inv \E[Y, (1, X)'].\\
\intertext{Let's build the matrices we need to perform this calculation.}
\E \left[\begin{bmatrix}1\\ X \end{bmatrix}[1, X]\right] &= \begin{bmatrix}1 & \E[X] \\ \E[X] & \E[X^2]\end{bmatrix}\\
&= \begin{bmatrix}1 & \frac{1}{2} \\ \frac{1}{2} & \frac{1}{3}\end{bmatrix}\\
\intertext{The finite variance allows us to invert this matrix:}
E[(1, X)' (1, X)]\inv &= \begin{bmatrix}4 & -6\\ -6 & 12 \end{bmatrix}.\\
\intertext{Now, onto the next matrix. We're going to use the Law of Iterated Expectations for this one, since we know $\E[Y \mid X]$.}
\E[Y (1, X)'] &= \E \begin{bmatrix}\E[Y \mid X] \\ X \E [Y \mid X]\end{bmatrix}\\
&= \E \begin{bmatrix} X^2 \\ X^3 \end{bmatrix}\\
&= \begin{bmatrix}\frac{1}{3} \\ \frac{1}{4}\end{bmatrix}.\\
\intertext{We can plug these matrices into our FOC formula:}
\beta_0 &= \E[(1, X)'(1, X)]\inv \E[Y, (1, X)']\	\
&= \begin{bmatrix}4 & -6 \\ -6 & 12 \end{bmatrix}\begin{bmatrix}\frac{1}{3}\\ \frac{1}{4}\end{bmatrix}\\ &=\begin{bmatrix}- \frac{1}{6} \\ 1\end{bmatrix}
\end{align*}

\newpage
\section{Question 2}
\label{sec:org62a4419}
  \begin{align*}
\nabla \E [\E [Y \mid X] &= \left(\frac{\partial \E[Y \mid X]}{\partial X_1} \ldots \frac{\partial \E [Y \mid X]}{\partial X_d}\right)'\\
\E [\|\nabla \E[Y \mid X] - b\|^2]&= \E \left[\sum^d_{i = 1}\left(\frac{\partial \E [Y \mid X]}{\partial X_i} - b_i\right)^2\right]\\
\frac{\partial}{\partial b_i} \E[\| \nabla \E[Y \mid X] - b\|^2] &= -2 \left[\frac{\partial \E[Y\mid X]}{\partial x_i} - b_i\right]\\
&= 0\\
\implies b^* &= \E\left[\frac{\partial \E [Y \mid X]}{\partial X_i}\right]
\intertext{This expression is the same as}
b_0 &= \nabla \E [\E [Y \mid X]].
  \end{align*}
For a counter example, consider \(X \sim U[0, 2]\); \(Y = X^3\). The OLS coefficient we get from this is
\begin{align*}
\frac{\E[(X^3 - \E(X^3))(X -  1)]}{\E[(X - 1)^2]} &= \frac{\E[(X^3 - 2) (X - 1)]}{\E[(X - 1)^2]}\\
&= \frac{9}{10},\\
\intertext{but}
E[3X^2] &= \int^2_0 3x^2 \cdot \frac{1}{2} dx \\
&= 4
\end{align*}

\newpage
\section{Question 3}
\label{sec:org0342841}
  \begin{enumerate}[label=\alph*)]
\item 
\begin{align*}
\E[Y_i \mid D_i] &= \E[D_i Y_i(1) + (1 - D_i)Y_i(0) \mid D_i]\\
&= D_i \E [Y_i \mid D_i = 1] + (1 - D_i) \E [Y_i(0) \mid D_i = 0]\\
&= \E[Y_i \mid D_i = 0] + D_i (\E [Y_i \mid D_i = 1] - \E[Y_i (0) \mid D_i = 0])\\
&= \alpha_0 + D_i \beta_0     \\
\intertext{Now define}
\eta &= Y_i - \alpha_0 - D_i \beta_0 \\
&= Y_i - \E[Y_i \mid D_i]\\
\E[\eta \mid D_i] &= \E[Y_i - \alpha_0 - D_i \beta_0]\\
&= \E [Y_i \mid D_i] - \alpha_0 - D_i \beta_0\\
&= \alpha_o + D_i \beta_0 - \alpha_0 - D_i \beta_0\\
&= 0
\end{align*}
\item
\begin{align*}
\beta_0 &= \E [Y_i(1) \mid D_i = 1] - \E [Y_i(0) \mid D_i = 0]\\
&= \E[Y_i(1) \mid D_i = 1] - \E[Y_i(0) \mid D_i = 0] + \E[Y_i(0)\mid D_i = 0] - \E[Y_i(0)\mid D_i = 0]\\
&= \E[Y_i - Y_i(0) \mid D_i = 1] + \E[Y_i(0) \mid D_i = 1] - \E[Y_i (0) \mid D_i = 0]
\end{align*}
\item ATEU should be positive if college has a positive impact on earnings.

\item Selection bias should be positive. Regardless of whetehr they attended college, more talented individuals would have earned more, so we are conflating the effect of attending college with these individuals' innate abilities.

\item OLS is not consistent for ATE regardless of heterogeneity, because we will still have a bias term. Note that, even with heterogeneity, we are only trying to identify the \emph{average} treatment effect.
  \end{enumerate}

\newpage
\section{Question 6}
\label{sec:org30a0a10}
For this problem, it's easier to work with the demeaned data, 
\[
  \tilde{\beta}_n = \arg \min_{b \in \R^d} \frac{1}{n} \sum^n_{i = 1} ((Y_i - \overline{Y}_n) - (X_i - \overline{X}_n)' b)^2.
  \]
\begin{align*}
\intertext{Starting with the forward direction,}
R^2 = 1 & \implies RSS = 0\\
\equiv 0 &= \sum^n_{i = 1} ((Y_i - \overline{Y}_n) - (X_i - \overline{X}_n)' \tilde{\beta}_n)^2\\
0 &= Y_i - \overline{Y}_n - (X_i - \overline{X}_n)' \tilde{\beta_n}\ \forall i\\
Y_i &= \underbrace{\overline{Y}_n - \overline{X}_n ' \tilde{\beta}_n}_{\alpha_0} + X_i' \underbrace{\tilde{\beta}_n}{b_0} \forall i\\
\intertext{To go the other way, suppose}
Y_i = a_0 + X_i' b_0 \ \forall i\\
\tilde{\beta}_n &= \arg \min_{b \in \R^d} \frac{1}{n} \sum^n_{i = 1} ((Y_i - \overline{Y}_n) - (X_i - \overline{X}_n)' b)^2\\
&= \arg \min_{b \in \R^d} \frac{1}{n} \sum^n_{i = 1} ((X_i - \overline{X}_n)' b_0 - (X_i - \overline{X}_n)' b)^2.\\
\intertext{The $\arg \min$ for this expression is $b = b_0$.}
Y_i - \overline{Y}_n &= a_0 - a_0 + (X_i - \overline{X}_n)' b_0\\
&= (X_i - \overline{X_n})' b_0\\
&= (X_i - \overline{X_n})' \tilde{\beta}_n\\
\implies RSS &= \sum^n_{i = 1} ((Y_i - \overline{Y}_n) - (X_i - \overline{X}_n)' \tilde{\beta}_n)^2 = 0\\
\implies R^2 &= 1
\end{align*}
\newpage
\section{Question 8}
\label{sec:org0a46612}
  \begin{enumerate}[label=\alph*)]
\item See the \verb|python| code below; the function that does this part of the problem is \verb|drop_missing_observations|.
\item The function that performs these calculations is \verb|calculate_summary_statistics|. Our dataset contains 2620.0 boys.                     2960 students were assigned to tracking schools.                     The average baseline original score was 0.028842416616841626,                     and our dataset contains 108 unique schools.


\item See the code below for the actual calculations; the code contains the outcome and covariates for each specification I report.

\newpage
\begin{table}
\caption{Regression to estimate the treatment effect, run on the sample of only girls}
\begin{center}
\begin{tabular}{lclc}
\toprule
\textbf{Dep. Variable:}    &    totalscore    & \textbf{  R-squared:         } &     0.005   \\
\textbf{Model:}            &       OLS        & \textbf{  Adj. R-squared:    } &     0.004   \\
\textbf{Method:}           &  Least Squares   & \textbf{  F-statistic:       } &     12.36   \\
\textbf{Date:}             & Mon, 08 Feb 2021 & \textbf{  Prob (F-statistic):} &  0.000446   \\
\textbf{Time:}             &     17:02:55     & \textbf{  Log-Likelihood:    } &   -3674.1   \\
\textbf{No. Observations:} &        2530      & \textbf{  AIC:               } &     7352.   \\
\textbf{Df Residuals:}     &        2528      & \textbf{  BIC:               } &     7364.   \\
\textbf{Df Model:}         &           1      & \textbf{                     } &             \\
\bottomrule
\end{tabular}
\begin{tabular}{lcccccc}
                  & \textbf{coef} & \textbf{std err} & \textbf{t} & \textbf{P$> |$t$|$} & \textbf{[0.025} & \textbf{0.975]}  \\
\midrule
\textbf{const}    &       0.0623  &        0.032     &     1.945  &         0.052        &       -0.001    &        0.125     \\
\textbf{tracking} &       0.1469  &        0.042     &     3.516  &         0.000        &        0.065    &        0.229     \\
\bottomrule
\end{tabular}
\begin{tabular}{lclc}
\textbf{Omnibus:}       & 185.750 & \textbf{  Durbin-Watson:     } &    1.427  \\
\textbf{Prob(Omnibus):} &   0.000 & \textbf{  Jarque-Bera (JB):  } &  172.304  \\
\textbf{Skew:}          &   0.576 & \textbf{  Prob(JB):          } & 3.84e-38  \\
\textbf{Kurtosis:}      &   2.447 & \textbf{  Cond. No.          } &     2.88  \\
\bottomrule
\end{tabular}
%\caption{OLS Regression Results}
\end{center}

Notes: \newline
 [1] Standard Errors assume that the covariance matrix of the errors is correctly specified.

\end{table}

\item

\begin{table}
\caption{Regression to estimate the treatment effect, run on the sample of only boys}
\begin{center}
\begin{tabular}{lclc}
\toprule
\textbf{Dep. Variable:}    &    totalscore    & \textbf{  R-squared:         } &     0.003   \\
\textbf{Model:}            &       OLS        & \textbf{  Adj. R-squared:    } &     0.002   \\
\textbf{Method:}           &  Least Squares   & \textbf{  F-statistic:       } &     7.538   \\
\textbf{Date:}             & Mon, 08 Feb 2021 & \textbf{  Prob (F-statistic):} &  0.00608    \\
\textbf{Time:}             &     17:02:55     & \textbf{  Log-Likelihood:    } &   -3671.1   \\
\textbf{No. Observations:} &        2620      & \textbf{  AIC:               } &     7346.   \\
\textbf{Df Residuals:}     &        2618      & \textbf{  BIC:               } &     7358.   \\
\textbf{Df Model:}         &           1      & \textbf{                     } &             \\
\bottomrule
\end{tabular}
\begin{tabular}{lcccccc}
                  & \textbf{coef} & \textbf{std err} & \textbf{t} & \textbf{P$> |$t$|$} & \textbf{[0.025} & \textbf{0.975]}  \\
\midrule
\textbf{const}    &      -0.0323  &        0.029     &    -1.115  &         0.265        &       -0.089    &        0.025     \\
\textbf{tracking} &       0.1063  &        0.039     &     2.746  &         0.006        &        0.030    &        0.182     \\
\bottomrule
\end{tabular}
\begin{tabular}{lclc}
\textbf{Omnibus:}       & 199.415 & \textbf{  Durbin-Watson:     } &    1.472  \\
\textbf{Prob(Omnibus):} &   0.000 & \textbf{  Jarque-Bera (JB):  } &  246.077  \\
\textbf{Skew:}          &   0.748 & \textbf{  Prob(JB):          } & 3.67e-54  \\
\textbf{Kurtosis:}      &   2.869 & \textbf{  Cond. No.          } &     2.79  \\
\bottomrule
\end{tabular}
%\caption{OLS Regression Results}
\end{center}

Notes: \newline
 [1] Standard Errors assume that the covariance matrix of the errors is correctly specified.

\end{table}

\item 

\newpage
\begin{table}
\caption{Regression to estimate the treatment effect for both boys and girls, run on the whole sample}
\begin{center}
\begin{tabular}{lclc}
\toprule
\textbf{Dep. Variable:}    &    totalscore    & \textbf{  R-squared:         } &     0.007   \\
\textbf{Model:}            &       OLS        & \textbf{  Adj. R-squared:    } &     0.007   \\
\textbf{Method:}           &  Least Squares   & \textbf{  F-statistic:       } &     12.92   \\
\textbf{Date:}             & Mon, 18 Jan 2021 & \textbf{  Prob (F-statistic):} &  2.09e-08   \\
\textbf{Time:}             &     22:59:22     & \textbf{  Log-Likelihood:    } &   -7348.6   \\
\textbf{No. Observations:} &        5150      & \textbf{  AIC:               } & 1.471e+04   \\
\textbf{Df Residuals:}     &        5146      & \textbf{  BIC:               } & 1.473e+04   \\
\textbf{Df Model:}         &           3      & \textbf{                     } &             \\
\bottomrule
\end{tabular}
\begin{tabular}{lcccccc}
                       & \textbf{coef} & \textbf{std err} & \textbf{t} & \textbf{P$> |$t$|$} & \textbf{[0.025} & \textbf{0.975]}  \\
\midrule
\textbf{const}         &      -0.0323  &        0.030     &    -1.086  &         0.277        &       -0.091    &        0.026     \\
\textbf{girl}          &       0.0946  &        0.043     &     2.193  &         0.028        &        0.010    &        0.179     \\
\textbf{treated\_boy}  &       0.1063  &        0.040     &     2.676  &         0.007        &        0.028    &        0.184     \\
\textbf{treated\_girl} &       0.1469  &        0.041     &     3.606  &         0.000        &        0.067    &        0.227     \\
\bottomrule
\end{tabular}
\begin{tabular}{lclc}
\textbf{Omnibus:}       & 351.267 & \textbf{  Durbin-Watson:     } &    1.399  \\
\textbf{Prob(Omnibus):} &   0.000 & \textbf{  Jarque-Bera (JB):  } &  397.937  \\
\textbf{Skew:}          &   0.658 & \textbf{  Prob(JB):          } & 3.88e-87  \\
\textbf{Kurtosis:}      &   2.647 & \textbf{  Cond. No.          } &     5.14  \\
\bottomrule
\end{tabular}
%\caption{OLS Regression Results}
\end{center}

Notes: \newline
 [1] Standard Errors assume that the covariance matrix of the errors is correctly specified.

\end{table}

\begin{align*}
Y_i &= \alpha_0 + \alpha_1 G_i + \beta_0 T_i \times (1 - G_i) + \beta_1 T_i \times G_i \\
\alpha_0 &= \text{ boy, untreated mean}\\
\alpha_1 &= \text{ girl, untreated mean}\\
\beta_0 &= \text{ boy, treatment effect}\\
\beta_1 &= \text{ girl, treatment effect}\\
\intertext{Let}
Y_i (1) &= \text{ outcome with treatment}\\
\E [Y_i \mid T_i, G_i] &= (1 - T_i) (1 - G_i) \E[Y_i^{\text{boy}}(0) \mid T_i = 0, \text{ boy}]\\
&\ + (1 - T_i) G_i \E[Y_i^{\text{girl}}(0) \mid T_i = 0, \text{ girl}]\\
&\ + T_i (1 - G_i) \E[Y_i^{\text{boy}}(1) \mid T_i = 1, \text{ boy}]\\
&\ + T_i G_i \E[Y_i^{\text{girl}}(1) \mid T_i = 1, \text{ girl}]\\
&= \underbrace{(1 - G_i) \E [Y_i^{\text{boy}(0)}]}_{\alpha_0} + \underbrace{ G_i \E [Y_i^{\text{girl}(0)}]}_{\alpha_1}\\
&\ + \underbrace{T_i (1 - G_i) \E [Y_i^{\text{boy}}(1) - Y_i^{\text{boy}}(0)]}_{\beta_0} + \underbrace{T_i G_i \E [Y_i^{\text{girl}}(1) - Y_i^{\text{girl}}(0)]}_{\beta_1}\\
\intertext{We can estimate these objects via OLS since conditional expectation is linear.}
\end{align*}

\item 

\begin{table}
\caption{Regression to estimate the treatment effect, run on the top half of the sample}
\begin{center}
\begin{tabular}{lclc}
\toprule
\textbf{Dep. Variable:}    &    totalscore    & \textbf{  R-squared:         } &     0.006   \\
\textbf{Model:}            &       OLS        & \textbf{  Adj. R-squared:    } &     0.005   \\
\textbf{Method:}           &  Least Squares   & \textbf{  F-statistic:       } &     14.60   \\
\textbf{Date:}             & Mon, 18 Jan 2021 & \textbf{  Prob (F-statistic):} &  0.000136   \\
\textbf{Time:}             &     22:59:22     & \textbf{  Log-Likelihood:    } &   -3748.5   \\
\textbf{No. Observations:} &        2642      & \textbf{  AIC:               } &     7501.   \\
\textbf{Df Residuals:}     &        2640      & \textbf{  BIC:               } &     7513.   \\
\textbf{Df Model:}         &           1      & \textbf{                     } &             \\
\bottomrule
\end{tabular}
\begin{tabular}{lcccccc}
                  & \textbf{coef} & \textbf{std err} & \textbf{t} & \textbf{P$> |$t$|$} & \textbf{[0.025} & \textbf{0.975]}  \\
\midrule
\textbf{const}    &       0.3882  &        0.030     &    13.133  &         0.000        &        0.330    &        0.446     \\
\textbf{tracking} &       0.1501  &        0.039     &     3.821  &         0.000        &        0.073    &        0.227     \\
\bottomrule
\end{tabular}
\begin{tabular}{lclc}
\textbf{Omnibus:}       & 200.720 & \textbf{  Durbin-Watson:     } &    1.468  \\
\textbf{Prob(Omnibus):} &   0.000 & \textbf{  Jarque-Bera (JB):  } &  116.240  \\
\textbf{Skew:}          &   0.372 & \textbf{  Prob(JB):          } & 5.74e-26  \\
\textbf{Kurtosis:}      &   2.291 & \textbf{  Cond. No.          } &     2.80  \\
\bottomrule
\end{tabular}
%\caption{OLS Regression Results}
\end{center}

Notes: \newline
 [1] Standard Errors assume that the covariance matrix of the errors is correctly specified.

\end{table}

\begin{table}
\caption{Regression to estimate the treatment effect, run on the bottom half of the sample}
\begin{center}
\begin{tabular}{lclc}
\toprule
\textbf{Dep. Variable:}    &    totalscore    & \textbf{  R-squared:         } &     0.006   \\
\textbf{Model:}            &       OLS        & \textbf{  Adj. R-squared:    } &     0.006   \\
\textbf{Method:}           &  Least Squares   & \textbf{  F-statistic:       } &     15.49   \\
\textbf{Date:}             & Mon, 18 Jan 2021 & \textbf{  Prob (F-statistic):} &  8.53e-05   \\
\textbf{Time:}             &     22:59:22     & \textbf{  Log-Likelihood:    } &   -3139.7   \\
\textbf{No. Observations:} &        2508      & \textbf{  AIC:               } &     6283.   \\
\textbf{Df Residuals:}     &        2506      & \textbf{  BIC:               } &     6295.   \\
\textbf{Df Model:}         &           1      & \textbf{                     } &             \\
\bottomrule
\end{tabular}
\begin{tabular}{lcccccc}
                  & \textbf{coef} & \textbf{std err} & \textbf{t} & \textbf{P$> |$t$|$} & \textbf{[0.025} & \textbf{0.975]}  \\
\midrule
\textbf{const}    &      -0.3987  &        0.026     &   -15.225  &         0.000        &       -0.450    &       -0.347     \\
\textbf{tracking} &       0.1349  &        0.034     &     3.935  &         0.000        &        0.068    &        0.202     \\
\bottomrule
\end{tabular}
\begin{tabular}{lclc}
\textbf{Omnibus:}       & 387.199 & \textbf{  Durbin-Watson:     } &     1.504  \\
\textbf{Prob(Omnibus):} &   0.000 & \textbf{  Jarque-Bera (JB):  } &   588.956  \\
\textbf{Skew:}          &   1.101 & \textbf{  Prob(JB):          } & 1.29e-128  \\
\textbf{Kurtosis:}      &   3.885 & \textbf{  Cond. No.          } &      2.86  \\
\bottomrule
\end{tabular}
%\caption{OLS Regression Results}
\end{center}

Notes: \newline
 [1] Standard Errors assume that the covariance matrix of the errors is correctly specified.

\end{table}

Based on the point estimates, students in the top half of the sample benefit more from being assigned to a tracking school.
  \end{enumerate}
\newpage
\inputminted{python}{ddk.py}
\end{document}